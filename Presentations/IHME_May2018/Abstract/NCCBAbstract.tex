\documentclass[11pt]{amsart}
\usepackage{geometry}                
\geometry{letterpaper}                  
\usepackage{graphicx}
\usepackage{hyperref}
\usepackage{amssymb}
\usepackage{todonotes}
\usepackage{epstopdf}

\usepackage{color}
\usepackage{soul}

%\DeclareGraphicsRule{.tif}{png}{.png}{`convert #1 `dirname #1`/`basename #1 .tif`.png}

<<<<<<< HEAD
\title{MGDrivE:  A Mosquito Population Framework to Evaluate and Optimize Gene-Drive Releases for Vector-Borne Diseases Control}
\author{H\'{e}ctor M. S\'{a}nchez C., Sean L. Wu, Jared Bennett, John M. Marshall}
=======
\title{MGDrivE: A Mosquito Population Framework to Evaluate and Optimize Releases of Gene-Drive Interventions for Vector-Borne Diseases Control}
\author{H\'{e}ctor M. S\'{a}nchez C., Sean L. Wu, Jared B. Bennett, John M. Marshall}
>>>>>>> ee7d51e4dccb57a73a078f1e2b994a96abd32865
\date{\today}                                           

\begin{document}
\maketitle

%The recent development of a CRISPR-Cas9-based homing system for the suppression of \emph{Anopheles gambiae} populations has raised a lot of interest in the mosquito-borne disease elimination community; as it opens the door to new possibilities and alternatives to traditional mosquito control interventions. This interest, however, generates a new, pressing need: to calculate impact estimates of these genetic drives. More specifically, the need for a fast, flexible, framework in which different versions of gene drives can be tested and evaluated (as new ideas and inheritance mechanisms are being developed with increasing frequency). \todo{IMPROVE THIS}

<<<<<<< HEAD
Recent developments of CRISPR-Cas9 based homing endonuclease gene drive systems for the suppression or replacement of mosquito populations have generated much interest in their use for control of mosquito-borne diseases (such as dengue, malaria, chikungunya and Zika). This is because genetic control of pathogen transmission may complement or even substitute traditional vector-control interventions,  which have had limited success in bringing the spread of these diseases to a halt. Despite excitement for the use of gene drives for mosquito control,  current modeling efforts have analyzed only a handful of these new approaches (usually studying just one per framework). Moreover, these models usually consider well-mixed populations with no explicit spatial dynamics. To this end, we are developing MGDrivE (Mosquito Gene DRIVe Explorer), in cooperation with the ``UCI Malaria Elimination Initiative'', as a flexible modeling framework to evaluate a variety of drive systems in spatial networks of mosquito populations. This framework provides a reliable testbed to evaluate and optimize the efficacy of gene drive mosquito releases. What separates MGDrivE from other models is the incorporation of mathematical and computational mechanisms to simulate a wide array of inheritance-based technologies within the same, coherent set of equations. We do this by treating the population dynamics, genetic inheritance operations, and migration between habitats as separate processes coupled together through the use of mathematical tensor operations. This way we can conveniently swap inheritance patterns whilst still making use of the same set of population dynamics equations.  This is a crucial advantage of our system, as it allows other research groups to test their ideas without developing new models and without the need to spend time adapting other frameworks to suit their needs.

=======
Recent developments of {\color{red} \st{a}} CRISPR-Cas9 based homing endonuclease gene drive system{\color{green} s} for the suppression or replacement of mosquito populations have generated much interest in their use for control of mosquito-borne diseases (such as dengue, malaria and {\color{green} Z}ika). This is because genetic control of pathogen transmission may complement or even substitute traditional vector-control interventions,  which have had limited success in bringing the spread of these diseases to a halt.
\\ \\ \\
{\color{green} Potential new first few lines? \\ \\ Traditional vector-control interventions, such as insecticide-treated nets and indoor residual spraying, have had limited success in halting the spread of malaria, dengue fever, or Zika virus. Recent developments in CRISPR-Cas9 gene drive systems provide an advanced method for the suppression or replacement of mosquito populations, thereby controlling the spread of vector-borne diseases. } 
\\ \\ \\
Despite excitement for the use of gene drive systems {\color{red} \st{for mosquito control}}{\color{green} to control mosquito-borne diseases}, modeling efforts to date have largely focused on {\color{red} \st{a}} single drive system{\color{green} s}, usually in well-mixed populations and without explicit spatial dynamics. To this end, we are developing MGDrivE (Mosquito Gene D{\color{green} RIV}e Explorer) in cooperation with the UCI Malaria Elimination Initiative{\color{red} \st{;}} as a flexible modeling framework to evaluate a variety of drive systems in spatial networks of mosquito populations. This framework {\color{red} \st{is aimed towards}} provid{\color{red} \st{ing}}{\color{green} es} a reliable testbed {\color{red} \st{where the spatial spread of gene drive mosquito releases can be evaluated and optimized}} {\color{green} to evaluate and optimize the efficacy of gene drive mosquito releases}. What separates MGDrivE from other models is the incorporation of mathematical and computational mechanisms {\color{red} \st{that allow us}} to simulate a wide array {\color{red} \st{number}} of inheritance-based technologies within the same, coherent set of equations. We do this by treating the population dynamics, genetic inheritance operations, and migration between habitats as separate processes{\color{red} \st{that are}} coupled together through the use of mathematical tensor operations. This way we can conveniently swap inheritance patterns whilst still using the same set of population dynamics equations and migratory processes (or modularly change them as needed). {\color{red} \st{We expect this to be}}{\color{green} This is} a crucial advantage of our system, as it {\color{red} \st{would}} allow{\color{green} s} other research groups to test their ideas without {\color{red} \st{the need to}} develop{\color{green} ing} new models {\color{red} \st{and without needing to spend}} {\color{green} or wasting} time adapting other frameworks to suit their needs.
>>>>>>> ee7d51e4dccb57a73a078f1e2b994a96abd32865
%By doing so, we are able to incorporate newly developed gene-drives, and simulate large-scale spatial scenarios without the need to make modifications to the core framework. We expect this to be a crucial advantage of our system, as it would allow other research groups to test their ideas without the need to develop new models and without needing to spend time adapting other frameworks to suit their needs.



%With this in mind, our research group is developing MGDrivE (Mosquito Gene Drive Explorer) as part of the UCI Malaria Elimination Initiative. This framework is aimed towards providing a reliable testbed where the spatial spread of gene drive mosquito releases can be evaluated and optimized. What separates MGDrivE from other models is the incorporation of mathematical and computational mechanisms that allow us to simulate a wide array number of inheritance-based technologies within the same, coherent set of equations. We do this by treating the population dynamics, genetic inheritance operations, and migration between habitats as separate processes that are coupled together through the use of mathematical tensor operations. This way we can conveniently swap inheritance patterns whilst still using the same set of population dynamics equations and migratory processes (or change them as needed). By doing so, we are able to incorporate newly developed gene-drives, and simulate large-scale spatial scenarios without the need to make modifications to the core framework. We expect this to be a crucial advantage of our system, as it would allow other research groups to test their ideas without the need to develop new models and without needing to spend time adapting other frameworks to suit their needs.

%\todo{ADD CONCLUSION HERE}

%https://www.youtube.com/playlist?list=PLRzY6w7pvIWqFJi94ZfhPkSVnazlUylpN
%https://chipdelmal.github.io/MGDrivE/}{https://chipdelmal.github.io/MGDrivE/

\end{document}  